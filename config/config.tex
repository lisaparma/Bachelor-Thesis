%**************************************************************
% file contenente le impostazioni della tesi
%**************************************************************

%**************************************************************
% Frontespizio
%**************************************************************
\newcommand{\myName}{Lisa Parma}                                    % autore
\newcommand{\myTitle}{Teamwork: una applicazione mobile per Zextras Suite}
\newcommand{\mySupervisor}{Francesco Ranzato}                        % relatore
\newcommand{\myCoSupervisor}{Nome Cognome}                      % co-relatore
% decomment line 12 or 13
\newbool{ifCoSupervisor}
% \booltrue{ifCoSupervisor}
\boolfalse{ifCoSupervisor}
\newcommand{\myDegree}{Tesi di laurea triennale}                % tipo di tesi
\newcommand{\myUni}{Università degli Studi di Padova}           % università
\newcommand{\myFaculty}{Corso di Laurea in Informatica}         % facoltà
\newcommand{\myDepartment}{Dipartimento di Matematica}          % dipartimento
\newcommand{\myLocation}{Padova}                                % dove
\newcommand{\myAA}{2017-2018}                                   % anno
\newcommand{\myTime}{Sett 2018}                                  % quando


%**************************************************************
% Impostazioni di impaginazione
% see: http://wwwcdf.pd.infn.it/AppuntiLinux/a2547.htm
%**************************************************************

\setlength{\parindent}{14pt}   % larghezza rientro della prima riga
\setlength{\parskip}{0pt}   % distanza tra i paragrafi


%**************************************************************
% Impostazioni di biblatex
%**************************************************************
\bibliography{res/bibliography} % database di biblatex

\defbibheading{bibliography}
{
    \cleardoublepage
    \phantomsection
    \addcontentsline{toc}{chapter}{\bibname}
    \chapter*{\bibname\markboth{\bibname}{\bibname}}
}

\setlength\bibitemsep{1.5\itemsep} % spazio tra entry

\DeclareBibliographyCategory{opere}
\DeclareBibliographyCategory{web}

\addtocategory{web}{site:agile-manifesto}
\addtocategory{web}{site:aws-lambda}
\addtocategory{web}{site:aws-api-gateway}
\addtocategory{web}{site:apex}
\addtocategory{web}{site:swagger}
\addtocategory{web}{site:messenger-facebook}
\addtocategory{web}{site:aws-dynamodb}
\addtocategory{web}{site:mocha}

\defbibheading{web}{\section*{Siti Web consultati}}


%**************************************************************
% Impostazioni di caption
%**************************************************************
\captionsetup{
    tableposition=top,
    figureposition=bottom,
    font=small,
    format=hang,
    labelfont=bf
}

%**************************************************************
% Impostazioni di glossaries
%**************************************************************
%% --------------------------------------------------------------------------- %
% Contrassegnare con \gls{} le prime ricorrenze delle parole qui riportate
% https://en.wikibooks.org/wiki/LaTeX/Glossary
%% --------------------------------------------------------------------------- %

\newglossaryentry{crossplatform} {
  name = crossplatform,
  description={
		Applicazione software, linguaggio di programmazione o un dispositivo hardware che funziona su più di un sistema o piattaforma}
}

\newglossaryentry{React Native} {
  name = React Native,
  description={
		Libreria JavaScript utilizzata per la realizzazione di interfacce utente per applicazioni mobili}
}
\newglossaryentry{open source} {
  name = open source,
  description={
		Movimento per cui gli autori di un software o i detentori dei suoi diritti rendono pubblico il codice sorgente, favorendone il libero studio e permettendo a programmatori indipendenti di apportarvi modifiche ed estensioni}
}
\newglossaryentry{storage} {
  name = storage,
  description={
		Dispositivi hardware, supporti per la memorizzazione, infrastrutture e software dedicati alla memorizzazione non volatile di grandi quantità di informazioni in formato elettronico}
}
\newglossaryentry{backup} {
  name = backup,
  description={
		Operazione di salvataggio su una qualunque memoria di massa dei dati di un utente archiviati nella memoria di un computer}
}
\newglossaryentry{agile} {
  name = agile,
  description = {
		Modelli di ciclo di vita del software che nascono come reazione alla troppa rigidità degli altri modelli, suntano sul soddisfacimento del cliente piuttosto che sulla pianificazione e qualità del software}
}
\newglossaryentry{processi} {
  name = processo,
  description={
		Insieme di attività collegate tra loro che trasformano ingressi in uscite secondo regole fissate e tramite risorse limitate}
}
\newglossaryentry{stakeholder} {
  name = stakeholder,
  description={
  		Soggetto che possiede un'interesse nei confronti di un progetto e che può influenzarne l'attività}
}

\newglossaryentry{requisiti} {
  name = requisito,
  description = {
		È una caratteristica che deve avere il software per soddisfare un certo bisogno del cliente. Descrive cosa il sistema deve fare, i servizi che offre e i vincoli sul suo funzionamento}
}

\newglossaryentry{release} {
  name = release,
  description={
		Specifica versione di un software resa disponibile ai suoi utenti finali}
}

\newglossaryentry{versionamento} {
  name = versionamento,
  description={
		Consiste nel tenere traccia degli stati che può assumere il progetto durante il suo sviluppo, diverso funzionalmente da un'altra versione}
}

\newglossaryentry{repository} {
  name = repository,
  description={
		Archivio nel quale sono raccolti e conservati dati e informazioni corredati da descrizioni (metadati) che li rendono identificabili dagli utenti}
}
\newglossaryentry{continuous integration} {
  name = continuous integration,
  description={
		Pratica di sviluppo che richiede agli sviluppatori di integrare il codice in un repository condiviso più volte al giorno. Ogni check-in viene quindi verificato da una build automatizzata, consentendo al team di rilevare i problemi in anticipo}
}

\newglossaryentry{deployment} {
  name = deployment,
  description={
		Rilascio al cliente di un sistema software o di un’applicazione.}
}
\newglossaryentry{stack trace} {
  name = stack trace,
  description={
		Elenco ordinato delle funzioni eseguite dal software}
}

\newglossaryentry{prototipo} {
  name = prototipo,
  description={
		Rappresenta la creazione di quello che sarà verosimilmente il prodotto finale}
}

\newglossaryentry{framework} {
  name = framework,
  description={
		Architettura logica di supporto (spesso un’implementazione logica di un particolare design pattern) su cui un software può essere costruito per facilitarne lo sviluppo da parte del programmatore}
}
\newglossaryentry{Android} {
  name = Android,
  description={
		sistema operativo per dispositivi mobili sviluppato da Google Inc. e basato sul kernel Linux. È stato progettato principalmente per smartphone e tablet}
}
\newglossaryentry{iOS} {
  name = iOS,
  description={
		Sistema operativo sviluppato da Apple per iPhone, iPod touch e iPad}
}
\newglossaryentry{Git} {
  name = git,
  description={
		Software per il controllo di versione distribuito con interfaccia a riga di comando sviluppato da Linus Torvalds nel 2005}
}
\newglossaryentry{guidelines} {
  name = guidelines,
  description={
		  Insieme di raccomandazioni sviluppate sistematicamente, sulla base di conoscenze continuamente aggiornate e valide, redatto allo scopo di rendere appropriato, e con un elevato standard di qualità, un comportamento desiderato}
}
\newglossaryentry{wireframe} {
  name = wireframe,
  description={
		Rappresenta il modello iniziale di rappresentazione di un sito web che ha lo scopo di identificare la struttura del sito web, l'architettura dell'informazione e la disposizione degli elementi nella pagina}
}
\newglossaryentry{xCode} {
  name= xCode,
  description={
		Ambiente di sviluppo integrato progettato da Apple contenente una suite di strumenti utili allo sviluppo di software per i sistemi macOS, iOS, watchOS e tvOS}
}
\newglossaryentry{Android Studio} {
  name = Android Studio,
  description={
		Ambiente di sviluppo integrato per lo sviluppo per la piattaforma Android distribuito sotto licenza Apache 2.0}
}
\newglossaryentry{analisi statica} {
  name= analisi statica,
  description={
		Tecnica di analisi che valuta il sistema o un suo componente basandosi sulla forma, sul contenuto e sulla struttura permettendo di individuare anomalie all'interno di documenti e codice sorgente}
}
\newglossaryentry{pattern architetturale} {
  name = pattern architetturale,
  description={
		Schemi di base per impostare l'organizzazione strutturale di un sistema software, si descrivono i sottosistemi predefiniti con i ruoli che essi assumono e le relazioni reciproche}
}

\newglossaryentry{unit testing} {
  name = unit testing,
  description={
		Attività di testing di singole unità software, il minimo componente di un programma dotato di funzionamento autonomo}
}

\newglossaryentry{architettura} {
  name = architettura,
  description={
		Organizzazione fondamentale di un sistema, definita dai suoi componenti, dalle relazioni reciproche tra i componenti e con l'ambiente, e i principi che ne governano la progettazione e l'evoluzione}
}
\newglossaryentry{bug} {
  name = bug,
  description={
		In italiano "baco", identifica un errore nella scrittura del codice sorgente che porta a comportamente anomali e non previsti del programmma. Un bug può essere introdotto anche in fase di compilazione o di progettazione del programma.}
}
\newglossaryentry{design pattern} {
  name = design pattern,
  description={
		Soluzione progettuale generale ad un problema ricorrente}
}
\newglossaryentry{mock} {
  name = mock,
  description={
		Oggetto simulato che riproduce il comportamento dell'oggetto reale in modo controllato}
}

% Acronimi
\newacronym{UML}{UML}{Unified Modeling Language}
\newacronym{UX}{UX}{User Experience}
\newacronym{IDE}{IDE}{Integrated Development Environment}
\newacronym{SDK}{SDK}{Software Development Kit}
\newacronym{SOAP}{SOAP}{Simple Object Access Protocol}
\newacronym{HTTP}{HTTP}{HyperText Transfer Protocol}
\newacronym{SMTP}{SMTP}{Simple Mail Transfer Protocol}
\newacronym{XML}{XML}{eXtensible Markup Language}
\newacronym{JSON}{JSON}{JavaScript Object Notation}
\newacronym{UI}{UI}{User Interface}
\newacronym{GUI}{GUI}{Grafical User Interface}



 % database di termini
\makeglossaries


%**************************************************************
% Impostazioni di graphicx
%**************************************************************
\graphicspath{{res/img/}} % cartella dove sono riposte le immagini


%**************************************************************
% Impostazioni di hyperref
%**************************************************************
\hypersetup{
    %hyperfootnotes=false,
    %pdfpagelabels,
    %draft,	% = elimina tutti i link (utile per stampe in bianco e nero)
    colorlinks=true,
    linktocpage=true,
    pdfstartpage=1,
    pdfstartview=FitV,
    % decommenta la riga seguente per avere link in nero (per esempio per la
%stampa in bianco e nero)
    %colorlinks=false, linktocpage=false, pdfborder={0 0 0}, pdfstartpage=1,
%pdfstartview=FitV,
    breaklinks=true,
    pdfpagemode=UseNone,
    pageanchor=true,
    pdfpagemode=UseOutlines,
    plainpages=false,
    bookmarksnumbered,
    bookmarksopen=true,
    bookmarksopenlevel=1,
    hypertexnames=true,
    pdfhighlight=/O,
    %nesting=true,
    %frenchlinks,
    urlcolor=webbrown,
    linkcolor=RoyalBlue,
    citecolor=webgreen,
    %pagecolor=RoyalBlue,
    %urlcolor=Black, linkcolor=Black, citecolor=Black, %pagecolor=Black,
    pdftitle={\myTitle},
    pdfauthor={\textcopyright\ \myName, \myUni, \myFaculty},
    pdfsubject={},
    pdfkeywords={},
    pdfcreator={pdfLaTeX},
    pdfproducer={LaTeX}
}

%**************************************************************
% Impostazioni di itemize
%**************************************************************
%\renewcommand{\labelitemi}{$\ast$}

%\renewcommand{\labelitemi}{$\bullet$}
%\renewcommand{\labelitemii}{$\cdot$}
%\renewcommand{\labelitemiii}{$\diamond$}
%\renewcommand{\labelitemiv}{$\ast$}


%**************************************************************
% Impostazioni di listings
%**************************************************************
\lstset{
    language=[LaTeX]Tex,%C++,
    keywordstyle=\color{RoyalBlue}, %\bfseries,
    basicstyle=\small\ttfamily,
    %identifierstyle=\color{NavyBlue},
    commentstyle=\color{Green}\ttfamily,
    stringstyle=\rmfamily,
    numbers=none, %left,%
    numberstyle=\scriptsize, %\tiny
    stepnumber=5,
    numbersep=8pt,
    showstringspaces=false,
    breaklines=true,
    frameround=ftff,
    frame=single
}


%**************************************************************
% Impostazioni di xcolor
%**************************************************************
\definecolor{webgreen}{rgb}{0,.5,0}
\definecolor{webbrown}{rgb}{.6,0,0}


%**************************************************************
% Altro
%**************************************************************

\newcommand{\omissis}{[\dots\negthinspace]} % produce [...]

% eccezioni all'algoritmo di sillabazione
\hyphenation
{
    ma-cro-istru-zio-ne
    gi-ral-din
}

\newcommand{\sectionname}{sezione}
\addto\captionsitalian{\renewcommand{\figurename}{figura}
                       \renewcommand{\tablename}{tabella}}

\newcommand{\glsfirstoccur}{\ap{{[g]}}}

\newcommand{\intro}[1]{\emph{\textsf{#1}}}

%**************************************************************
% Environment per ``rischi''
%**************************************************************
\newcounter{riskcounter}                % define a counter
\setcounter{riskcounter}{0}             % set the counter to some initial value

%%%% Parameters
% #1: Title
\newenvironment{risk}[1]{
    \refstepcounter{riskcounter}        % increment counter
    \par \noindent                      % start new paragraph
    \textbf{\arabic{riskcounter}. #1}   % display the title before the
                                        % content of the environment is
%displayed
}{
    \par\medskip
}

\newcommand{\riskname}{Rischio}

\newcommand{\riskdescription}[1]{\textbf{\\Descrizione:} #1.}

\newcommand{\risksolution}[1]{\textbf{\\Soluzione:} #1.}

%**************************************************************
% Comandi  per i casi d'uso
%**************************************************************

\newcommand{\usecase}[7]{
	\subsection{UC #1 - #2}
	\begin{figure}[H] 
		\centering
		\includegraphics[scale=0.2]{UC/#1}
		\caption{UC #1 - #2}
	\end{figure}
	\begin{center}
		\bgroup
		\def\arraystretch{1.8}     
		\begin{longtable}{  p{4cm} | p{9.5cm} } 
			\textbf{Attori Primari} & #3 \\ 
			\textbf{Descrizione} & .#4 \\ 
			\textbf{Precondizioni}  & #5 \\
			\textbf{Postcondizioni} & #6  \\ 
			\textbf{Flusso Principale} & #7 
		\end{longtable}
		\egroup
	\end{center}
}
%**************************************************************
% Comandi per il fix
%**************************************************************
\newcommand{\note}[1]{
	\todo[inline, backgroundcolor=orange!25,bordercolor=orange!]{#1}
}

\newcommand{\unsure}[1]{
	\texthl{#1}
}

\newcommand{\mmh}[2]{
	\texthl{#1}
	\todo[backgroundcolor=red!10]{#2}
}

\newcommand{\postit}[1]{
	\todo[noline, backgroundcolor=yellow!20]{#1}
}

