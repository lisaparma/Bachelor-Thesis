\section{UCG - Caso d'uso generale}
	\begin{figure}[H] 
		\centering
		\includegraphics[scale=0.2]{UC/UCG}
		\caption{UCG - Caso d'uso generale}
	\end{figure}
	\begin{center}
	\bgroup
	\def\arraystretch{1.8}     
	\begin{longtable}{  p{4cm} | p{9.5cm} } 
		\textbf{Attori Primari} & Utente Zimbra, utente OpenChat e utente Teamwork web \\ 
		\textbf{Descrizione} & L'utente, se dispone della zimlet OpenChat potrà interagire con l'applicazione avendo modo di gestire e contattare la sua buddylist. Se nei suoi server Zimbra dispone della zimlet Teamwork web potrà interagire anche con i gruppi di cui fa parte. \\ 
		\textbf{Precondizioni}  & Il sistema è avviato e connesso a dei server Zimbra pronti per l'autenticazione; \\
		\textbf{Postcondizioni} & Il sistema ha eseguito tutte le funzioni richieste dall' attore permettendogli così di interagire con la sua buddylist e con i gruppi di cui fa parte  \\ 
		\textbf{Scenario principale} & 
		1. L’utente Zimbra vuole accedere all'applicazione Teamwork per mobile per gestire e comunicare con la sua lista contatti e con i gruppi;\newline
		2. L’utente OpenChat ha a disposizione la buddylist con le relative chat in modo da gestire nel modo che crede opportuno la comunicazione con i suoi buddy.
	\end{longtable}
	\egroup
\end{center}

% ---------------------------------------------------------------------------- %
\section{UC1 - Accesso}
	\begin{figure}[H] 
		\centering
		\includegraphics[scale=0.17]{UC/UC1}
		\caption{UC1 - Accesso}
	\end{figure}
	\begin{center}
	\bgroup
	\def\arraystretch{1.8}     
	\begin{longtable}{  p{4cm} | p{9.5cm} } 
		\textbf{Attori Primari} & Utente Zimbra; \\ 
		\textbf{Descrizione} & L’attore si può autenticare inserendo l'e-mail e la password con cui è registrato al server Zimbra; \\ 
		\textbf{Precondizioni}  & Il sistema è avviato, pronto per l’utilizzo e mostra la pagina di login; \\
		\textbf{Postcondizioni} & Il sistema ha autenticato l’attore e gli mostra la sua area riservata;  \\ 
		\textbf{Scenario principale} & 
			1. L’attore inserisce l'e-mail con cui è registrato al server Zimbra); \newline
			2. L’attore inserisce la sua password; \newline
			3. L’attore conferma il login e accede alla sua area riservata.\\
		\textbf{Estensioni} & Visualizzazione errore login (\ref{UC2}).
	\end{longtable}
	\egroup
\end{center}

\section{UC1.1 - Inserimento email}
	\begin{center}
	\bgroup
	\def\arraystretch{1.8}     
	\begin{longtable}{  p{4cm} | p{9.5cm} } 
		\textbf{Attori Primari} & Utente Zimbra; \\ 
		\textbf{Descrizione} & L’attore si può autenticare inserendo l'e-mail con cui è registrato al server Zimbra; \\ 
		\textbf{Precondizioni}  & Il sistema è avviato, pronto per l’utilizzo e mostra la pagina di login; \\
		\textbf{Postcondizioni} & Il sistema resta in attesa dell'inserimento della password da parte dell'attore per continuare l'autenticazione;  \\ 
		\textbf{Scenario principale} & 
		1. L’attore inserisce l'e-mail con cui è registrato al server Zimbra).
	\end{longtable}
	\egroup
\end{center}
\section{UC1.2 - Inserimento password}
	\begin{center}
	\bgroup
	\def\arraystretch{1.8}     
	\begin{longtable}{  p{4cm} | p{9.5cm} } 
		\textbf{Attori Primari} & Utente Zimbra; \\ 
		\textbf{Descrizione} & L’attore si può autenticare inserendo la password del profilo Zimbra; \\ 
		\textbf{Precondizioni}  & Il sistema è avviato, pronto per l’utilizzo e mostra la pagina di login; \\
		\textbf{Postcondizioni} & Il sistema resta in attesa della conferma da parte dell'attore per continuare l'autenticazione;  \\ 
		\textbf{Scenario principale} & 
		1. L’attore inserisce la sua password.
	\end{longtable}
	\egroup
\end{center}
% ---------------------------------------------------------------------------- %

\section{UC2 - Visualizzazione errore login} \label{UC2}
	\begin{center}
	\bgroup
	\def\arraystretch{1.8}     
	\begin{longtable}{  p{4cm} | p{9.5cm} } 
		\textbf{Attori Primari} & Utente Zimbra; \\ 
		\textbf{Descrizione} &  L'attore visualizza un messaggio di errore in quanto le credenziali da lui inserite non sono corrette; \\ 
		\textbf{Precondizioni}  & L'attore ha cercato di effettuare il login inserendo delle credenziali errate; \\
		\textbf{Postcondizioni} & L'attore ha visualizzato il messaggio di errore;  \\ 
		\textbf{Scenario principale} & 
		1. L'attore prova ad autenticarsi inserendo delle credenziali errate; \newline
		2. L'attore visualizza il messaggio di errore;
	\end{longtable}
	\egroup
\end{center}

% ---------------------------------------------------------------------------- %

\section{UC3 - Visualizzazione buddylist}
	\begin{figure}[H] 
	\centering
	\includegraphics[scale=0.17]{UC/UC3}
	\caption{UC3 - Visualizzazione buddylist}
\end{figure}
	\begin{center}
	\bgroup
	\def\arraystretch{1.8}     
	\begin{longtable}{  p{4cm} | p{9.5cm} } 
		\textbf{Attori Primari} & Utente OpenChat; \\ 
		\textbf{Descrizione} &  L'attore può visualizzare l'intera lista dei buddy presenti nella sua buddylist; \\ 
		\textbf{Precondizioni}  & Talkapp presenta all'attore una pagina contenente la lista dei suoi buddy; \\
		\textbf{Postcondizioni} & L'attore ha visualizzato la lista dei suoi buddy;  \\ 
		\textbf{Scenario principale} & 
		1. L'attore seleziona l'icona nella tab riguardante la buddylist; \newline
		2. L'attore visualizza la buddylist contenente tutti i suoi buddy;
	\end{longtable}
	\egroup
\end{center}

\section{UC3.1 - Visualizzazione card buddy}
	\begin{figure}[H] 
	\centering
	\includegraphics[scale=0.17]{UC/UC3_1}
	\caption{UC3.1 - Visualizzazione card buddy}
\end{figure}
		\begin{center}
		\bgroup
		\def\arraystretch{1.8}     
		\begin{longtable}{  p{4cm} | p{9.5cm} } 
			\textbf{Attori Primari} & Utente OpenChat; \\ 
			\textbf{Descrizione} &  L'attore può visualizzare dei dati significativi di ogni buddy presente nella sua buddylist; \\ 
			\textbf{Precondizioni}  & Talkapp presenta all'attore una pagina contenente la lista dei suoi buddy; \\
			\textbf{Postcondizioni} & L'attore ha visualizzato per ogni buddy una card contenente dei dettagli;  \\ 
			\textbf{Scenario principale} & 
			1. L'attore seleziona l'icona nella tab riguardante la buddylist; \newline
			2. L'attore visualizza per ogni buddy una card in cui sono presenti dei dettagli;
		\end{longtable}
		\egroup
	\end{center}

\section{UC3.1.1 - Visualizzazione nickname}
	\begin{center}
	\bgroup
	\def\arraystretch{1.8}     
	\begin{longtable}{  p{4cm} | p{9.5cm} } 
		\textbf{Attori Primari} & Utente OpenChat; \\ 
		\textbf{Descrizione} &  L'attore può visualizzare il nickname dato ad un buddy della sua buddylist; \\ 
		\textbf{Precondizioni}  & Talkapp presenta all'attore una pagina contenente la lista dei suoi buddy; \\
		\textbf{Postcondizioni} & L'attore ha visualizzato per ogni buddy il nickname dato;  \\ 
		\textbf{Scenario principale} & 
		1. L'attore seleziona l'icona nella tab riguardante la buddylist; \newline
		2. L'attore visualizza il nickname del buddy nella card a esso collegata;
	\end{longtable}
	\egroup
\end{center}
\section{UC3.1.2 - Visualizzazione immagine profilo}
	\begin{center}
	\bgroup
	\def\arraystretch{1.8}     
	\begin{longtable}{  p{4cm} | p{9.5cm} } 
		\textbf{Attori Primari} & Utente OpenChat; \\ 
		\textbf{Descrizione} &  L'attore può visualizzare l'immagine profilo di ogni buddy della sua buddylist; \\ 
		\textbf{Precondizioni}  & Talkapp presenta all'attore una pagina contenente la lista dei suoi buddy; \\
		\textbf{Postcondizioni} & L'attore ha visualizzato per ogni buddy  l'immagine profilo;  \\ 
		\textbf{Scenario principale} & 
		1. L'attore seleziona l'icona nella tab riguardante la buddylist; \newline
		2. L'attore visualizza l'immagine profilo del buddy nella card a esso collegata;
	\end{longtable}
	\egroup
\end{center}
\section{UC3.1.3 - Visualizzazione status}
	\begin{center}
		\bgroup
		\def\arraystretch{1.8}     
		\begin{longtable}{  p{4cm} | p{9.5cm} } 
			\textbf{Attori Primari} & Utente OpenChat; \\ 
			\textbf{Descrizione} &  L'attore può visualizzare lo status di ogni buddy della sua buddylist; \\ 
			\textbf{Precondizioni}  & Talkapp presenta all'attore una pagina contenente la lista dei suoi buddy; \\
			\textbf{Postcondizioni} & L'attore ha visualizzato per ogni buddy  il loro status;  \\ 
			\textbf{Scenario principale} & 
			1. L'attore seleziona l'icona nella tab riguardante la buddylist; \newline
			2. L'attore visualizza lo status del buddy nella card a esso collegata;
		\end{longtable}
		\egroup
	\end{center}
\section{UC3.1.4 - Visualizzazione data ultimo messaggio}
	\begin{center}
		\bgroup
		\def\arraystretch{1.8}     
		\begin{longtable}{  p{4cm} | p{9.5cm} } 
			\textbf{Attori Primari} & Utente OpenChat; \\ 
			\textbf{Descrizione} &  L'attore può visualizzare la data dell'ultimo messaggio scambiato con un particolare buddy della propria buddylist; \\ 
			\textbf{Precondizioni}  & Talkapp presenta all'attore una pagina contenente la lista dei suoi buddy; \\
			\textbf{Postcondizioni} & L'attore ha visualizzato per ogni buddy  la data dell'ultimo messaggio scambiato con esso;  \\ 
			\textbf{Scenario principale} & 
			1. L'attore seleziona l'icona nella tab riguardante la buddylist; \newline
			2. L'attore visualizza la data dell'ultimo messaggio scambiato con un buddy nella card a esso collegata;
		\end{longtable}
		\egroup
	\end{center}
\section{UC3.1.5 - Visualizzazione notifiche}
	\begin{center}
		\bgroup
		\def\arraystretch{1.8}     
		\begin{longtable}{  p{4cm} | p{9.5cm} } 
			\textbf{Attori Primari} & Utente OpenChat; \\ 
			\textbf{Descrizione} &  L'attore può visualizzare il numero dei messaggi ricevuti da un particolare buddy della sua buddylist e non ancora letti; \\ 
			\textbf{Precondizioni}  & Talkapp presenta all'attore una pagina contenente la lista dei suoi buddy; \\
			\textbf{Postcondizioni} & L'attore ha visualizzato per ogni buddy il numero dei messaggi ricevuti da esso e non ancora letti;  \\ 
			\textbf{Scenario principale} & 
			1. L'attore seleziona l'icona nella tab riguardante la buddylist; \newline
			2. L'attore visualizza il numero dei messaggi ricevuto da un particolare buddy e non ancora letti nella card a esso collegata;
		\end{longtable}
		\egroup
	\end{center}

% ---------------------------------------------------------------------------- %

\section{UC4 - Visualizzazione storico chat}
	\begin{figure}[H] 
	\centering
	\includegraphics[scale=0.17]{UC/UC4}
	\caption{UC4 - Visualizzazione storico chat}
\end{figure}
	\begin{center}
	\bgroup
	\def\arraystretch{1.8}     
	\begin{longtable}{  p{4cm} | p{9.5cm} } 
		\textbf{Attori Primari} & Utente OpenChat; \\ 
		\textbf{Descrizione} &  L'attore può visualizzare lo storico delle conversazioni avute con i suoi buddy; \\ 
		\textbf{Precondizioni}  & Talkapp presenta all'attore una pagina contenente la lista dei suoi buddy; \\
		\textbf{Postcondizioni} & L'attore ha visualizzato la lista dei messaggi scambiati con un particolare buddy;  \\ 
		\textbf{Scenario principale} & 
		1. L'attore seleziona un buddy dalla sua buddylist; \newline
		2. L'attore visualizza la lista dei messaggi scambiati nel tempo con il buddy selezionato;
	\end{longtable}
	\egroup
\end{center}

\section{UC4.1 - Visualizzazione messaggi ricevuti}
	\begin{center}
	\bgroup
	\def\arraystretch{1.8}     
	\begin{longtable}{  p{4cm} | p{9.5cm} } 
		\textbf{Attori Primari} & Utente OpenChat; \\ 
		\textbf{Descrizione} &  L'attore può visualizzare i messaggi ricevuti da un particolare buddy; \\ 
		\textbf{Precondizioni}  & Talkapp presenta all'attore una pagina contenente la lista dei suoi buddy; \\
		\textbf{Postcondizioni} & L'attore ha visualizzato la lista dei messaggi ricevuti  da un particolare buddy;  \\ 
		\textbf{Scenario principale} & 
		1. L'attore seleziona un buddy dalla sua buddylist; \newline
		2. L'attore visualizza la lista dei messaggi ricevuti nel tempo dal buddy selezionato;
	\end{longtable}
	\egroup
\end{center}

\section{UC4.2 - Visualizzazione messaggi inviati}
	\begin{center}
	\bgroup
	\def\arraystretch{1.8}     
	\begin{longtable}{  p{4cm} | p{9.5cm} } 
		\textbf{Attori Primari} & Utente OpenChat; \\ 
		\textbf{Descrizione} &  L'attore può visualizzare i messaggi inviati ad un particolare buddy; \\ 
		\textbf{Precondizioni}  & Talkapp presenta all'attore una pagina contenente la lista dei suoi buddy; \\
		\textbf{Postcondizioni} & L'attore ha visualizzato la lista dei messaggi inviati  ad un particolare buddy;  \\ 
		\textbf{Scenario principale} & 
		1. L'attore seleziona un buddy dalla sua buddylist; \newline
		2. L'attore visualizza la lista dei messaggi inviati nel tempo al buddy selezionato;
	\end{longtable}
	\egroup
\end{center}

% ---------------------------------------------------------------------------- %

\section{UC5 - Invio messaggi}
	\begin{center}
	\bgroup
	\def\arraystretch{1.8}     
	\begin{longtable}{  p{4cm} | p{9.5cm} } 
		\textbf{Attori Primari} & Utente OpenChat; \\ 
		\textbf{Descrizione} &  L'attore può inviare messaggi ad un buddy; \\ 
		\textbf{Precondizioni}  & Talkapp presenta all'attore una pagina contenente lo storico delle conversazioni avute con un particolare buddy; \\
		\textbf{Postcondizioni} & L'attore ha inviato un messaggio ad un buddy; \\ 
		\textbf{Scenario principale} & 
		1. L'attore seleziona un buddy dalla sua buddylist; \newline
		2. L'attore digita un messaggio nel box apposito; \newline
		3. L'attore preme il pulsante di fianco al box per confermare l'invio;
		4. Il messaggio appare nello storico della chat.
	\end{longtable}
	\egroup
\end{center}

\section{UC6 - Modifica buddylist}
	\begin{center}
	\bgroup
	\def\arraystretch{1.8}     
	\begin{longtable}{  p{4cm} | p{9.5cm} } 
		\textbf{Attori Primari} & Utente OpenChat; \\ 
		\textbf{Descrizione} &  L'attore può modificare la buddylist visualizzata; \\ 
		\textbf{Precondizioni}  & Talkapp presenta all'attore una pagina contenente la lista dei suoi buddy \\
		\textbf{Postcondizioni} & L'attore ha modificato la buddylist che viene visualizzata; \\ 
		\textbf{Scenario principale} & 
		1. L'attore seleziona un buddy da rimuovere o digita un contatto da aggiungere; \newline
		2. L'attore conferma di voler cambiare la buddylist tramite l''azione svolta;
	\end{longtable}
	\egroup
\end{center}

\section{UC7 - Aggiunta buddy}
	\begin{center}
	\bgroup
	\def\arraystretch{1.8}     
	\begin{longtable}{  p{4cm} | p{9.5cm} } 
		\textbf{Attori Primari} & Utente OpenChat; \\ 
		\textbf{Descrizione} &  L'attore può aggiungere un nuovo buddy alla sua buddylist; \\ 
		\textbf{Precondizioni}  & Talkapp presenta all'attore una pagina contenente un form per l'aggiunta di un buddy; \\
		\textbf{Postcondizioni} & L'attore ha aggiunto un buddy alla sua buddylist;  \\ 
		\textbf{Scenario principale} & 
		1. L'attore inserisce il nickname che vuole dare al buddy; \newline
		2. L'attore inserisce l'e-mail del buddy che vuole aggiungere; \newline
		3. L'attore conferma l'aggiunta del buddy alla propria buddylist.
	\end{longtable}
	\egroup
\end{center}

\section{UC8 - Rimozione buddy}
	\begin{center}
	\bgroup
	\def\arraystretch{1.8}     
	\begin{longtable}{  p{4cm} | p{9.5cm} } 
		\textbf{Attori Primari} & Utente OpenChat; \\ 
		\textbf{Descrizione} &  L'attore può rimuovere un buddy dalla sua buddylist; \\ 
		\textbf{Precondizioni}  & Talkapp presenta all'attore una pagina contenente i dettagli del buddy selezionato e le azioni che possono essere svolte; \\
		\textbf{Postcondizioni} & L'attore ha rimosso il buddy dalla sua buddylist;  \\ 
		\textbf{Scenario principale} & 
		1. L'attore seleziona il buddiy dalla buddylist che vuole rimuovere; \newline
		2. L'attore seleziona la voce "Rimozione" dal menù di azioni disponibili; \newline
		3. L'attore conferma la rimozione del buddy dalla propria buddylist.
	\end{longtable}
	\egroup
\end{center}

\section{UC9 - Ricerca buddy}
	\begin{center}
	\bgroup
	\def\arraystretch{1.8}     
	\begin{longtable}{  p{4cm} | p{9.5cm} } 
		\textbf{Attori Primari} & Utente OpenChat; \\ 
		\textbf{Descrizione} &  L'attore può ricercare un  buddy tra quelli presenti nella sua buddylist; \\ 
		\textbf{Precondizioni}  & Talkapp presenta all'attore una pagina contenente un campo di ricerca per ricercare un buddy in base ad un certo testo; \\
		\textbf{Postcondizioni} & L'attore ha ricercato un buddy dalla sa buddylist in base ad un testo da lui inserito;  \\ 
		\textbf{Scenario principale} & 
		1. L'attore inserisce dei caratteri nel campo di ricerca; \newline
		2. L'attore visualizza i buddy corrispondenti alla ricerca da lui effettuata;
	\end{longtable}
	\egroup
\end{center}

\section{UC10 - Modifica status}
	\begin{center}
	\bgroup
	\def\arraystretch{1.8}     
	\begin{longtable}{  p{4cm} | p{9.5cm} } 
		\textbf{Attori Primari} & Utente OpenChat; \\ 
		\textbf{Descrizione} &  L'attore può modificare lo status con cui viene visualizzato dagli altri utenti; \\ 
		\textbf{Precondizioni}  & Talkapp presenta all'attore una pagina contenente un elenco degli statusdis ponibili; \\
		\textbf{Postcondizioni} & L'attore ha modificato lo status con cui viene visualizzato dagli altri utenti; \\ 
		\textbf{Scenario principale} & 
		1. L'attore seleziona uno status tra quelli disponibili; \newline
		2. L'attore conferma di voler cambiate status con quello selezionato;
	\end{longtable}
	\egroup
\end{center}

\section{UC11 - Modifica tema}
	\begin{center}
	\bgroup
	\def\arraystretch{1.8}     
	\begin{longtable}{  p{4cm} | p{9.5cm} } 
		\textbf{Attori Primari} & Utente OpenChat; \\ 
		\textbf{Descrizione} &  L'attore può modificare il tema dei colori con cui visualizza l'applicazione; \\ 
		\textbf{Precondizioni}  & Talkapp presenta all'attore una pagina contenente un elenco con tutti i temi disponibili; \\
		\textbf{Postcondizioni} & L'attore ha modificato il tema dell'applicazione; \\ 
		\textbf{Scenario principale} & 
		1. L'attore seleziona un tema tra quelli disponibili da applicare \newline
		2. L'attore conferma di voler applicare quel tema;
	\end{longtable}
	\egroup
\end{center}

\section{UC12 - Visualizzazione gruppi}
	\begin{center}
	\bgroup
	\def\arraystretch{1.8}     
	\begin{longtable}{  p{4cm} | p{9.5cm} } 
		\textbf{Attori Primari} & Utente Teamwork web; \\ 
		\textbf{Descrizione} &  L'attore può visualizzare l'elenco dei gruppi di cui fa parte; \\ 
		\textbf{Precondizioni}  & Talkapp presenta una pagina contenente i gruppi di cui fa parte; \\
		\textbf{Postcondizioni} & L'attore ha visualizzato i gruppi di cui fa parte; \\ 
		\textbf{Scenario principale} & 
		1. L'attore seleziona l'icona nella tab riguardante i gruppi; \newline
		2. L'attore visualizza i gruppi di cui fa parte;
	\end{longtable}
	\egroup
\end{center}

