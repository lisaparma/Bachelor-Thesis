% Analisi dei requisiti

\chapter{Analisi dei requisiti}\label{chap:requirements}

\section{Attori coinvolti}
\gl{parola}{PParola}
\gl{ceo}{CEO}
\section{Casi d'uso principali}
Considerando l'obbiettivo del progetto spiegato nella sezione \ref{sec:intaz}, i casi d'uso (come i requisiti funzionali) sono stati soggetti a sviluppo durante tutto il corso del progetto. In questa sezione vengono riportati e descritti i casi d'uso principali che sono stati delineati fin da subito tramite le richieste dell'azienda e quelli di alto livello, in modo da poter dare una visione d’insieme più precisa del prodotto sviluppato.

Ogni caso d'uso identificato viene qui riportato attraverso un codice univoco gerarchico nella forma:
$$ \textbf{UC \{codice\_padre\}.\{codice\_figlio\}  } $$
\begin{itemize}
	\item Le prime due lettere identificano che si tratta di un caso d'uso;
	\item Il codice padre è un numero univoco che identifica un caso d'uso;
	\item Il codice figlio è un numero progressivo che identifica i sottocasi;\\
\end{itemize}


\usecase{1}{Accesso}{Attore}{Desc}{pre}{post}{flusso}

\section{Requisiti}
Ogni requisito qui riportato è identificato da un codice, ed è rappresentato nel seguente modo:
$$ \textbf{R \{importanza\}\{tipo\}\{numero\_vincolo\} } $$

\begin{itemize}
	\item La R sta per requisito ed è presente in tutti i codici;
	\item Il primo valore rappresenta l'importanza: 0 se il requisito è obbligatorio, 1 se è desiderabile, 2 per gli opzionali;
	\item Il terzo valore indica il tipo: F per i requisiti funzionali, Q per quelli di qualità, P se prestazionale, V se di vincolo;
	\item L'ultimo numero indica il numero del vincolo. La struttura numerica di quest'ultimo rispetta le stesse regole dei Casi D'Uso.
\end{itemize}

\subsection{Principali requisiti funzionali}
\begin{longtable}{|c|c|}
	\hline
	\textbf{Id Requisito} & \textbf{Descrizione}\\
	\hline
	\endhead
	R0F1 & ....  \\ \hline 
	\caption{Requisiti di qualità}
	\label{tabella:req}
\end{longtable}

\subsection{Requisiti di vincolo}
\begin{longtable}{|c|c|}
	\hline
	\textbf{Id Requisito} & \textbf{Descrizione}\\
	\hline
	\endhead
	R0V1 & ....  \\ \hline 
	\caption{Requisiti di vincolo}
	\label{tabella:reqV}
\end{longtable}

\subsection{Requisiti di qualità}
\begin{longtable}{|c|c|}
	\hline
	\textbf{Id Requisito} & \textbf{Descrizione}\\
	\hline
	\endhead
	R0Q1 & ....  \\ \hline 
	\caption{Requisiti di qualità}
	\label{tabella:reqQ}
\end{longtable}

