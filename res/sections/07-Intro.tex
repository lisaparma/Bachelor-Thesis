%% INIZIO CONTENUTO TESI
\mainmatter

\chapter{Introduzione}\label{chap:intro}

\section{Organizzazione dell'elaborato}
Questo elaborato è organizzato come segue: 
\begin{itemize}
	\item Capitolo~\ref{chap:intro}: viene descritta la struttura del documento e tutte le convenzioni adottate nella stesura dello stesso;
	\item Capitolo~\ref{chap:company}: viene introdotta l'azienda ospitante, i loro prodotti nel mercato e il metodo di lavoro seguito al suo interno;
	\item Capitolo~\ref{chap:project}: viene presentato il progetto svolto durante questo stage, gli obiettivi, il mio contributo ad esso come è stato pianificato il lavoro;
	\item Capitolo ~\ref{chap:tec}: vengono descritte le tecnologie che sono state studiate e adottate durante il mio stage;
	\item Capitolo~\ref{chap:requirements}: sono elencati i requisiti che l'azienda ha definito per il prodotto finale;
	\item Capitolo~\ref{chap:design}: sono spiegate le scelte implementative e di design dell'applicazione;
	\item Capitolo~\ref{chap:tests}: vengono descritte le fasi di verifica e validazione del prodotto che sono state svolte:
	\item Capitolo~\ref{chap:conclusions}: sono presenti delle mie considerazioni sia sull'applicazione prodotta, che sul lavoro svolto e il periodo di stage in generale.
\end{itemize}

\section{Convenzioni adottate}
Nella stesura del presente documento sono state adottate le seguenti convenzioni
tipografiche:
\begin{itemize}
	\item la prima occorrenza di termini tecnici, ambigui o di acronimi verrà marcata in corsivo	e con una g a pedice, in modo da indicare la sua presenza nel glossario riportato in fondo a questo documento;
	\item i termini tecnici in lingua inglese non traducibili e non presenti nel glossario, verranno evidenziati in \emph{corsivo};
	\item la prima occorrenza di un acronimo viene riportata con la dicitura estesa, in
	\emph{corsivo}, e le lettere che compongono l’acronimo vengono riportate in \textbf{grassetto}. Ogni successiva occorrenza presenterà solo la forma ridotta;
	\item le parole chiave presenti in ciascun paragrafo saranno marcate in \textbf{grassetto}, in modo da consentirne una facile individuazione;
	\item ogni termine corrispondente a nomi di file, componenti dell’architettura o codice verrà marcato utilizzando un font a \texttt{larghezza fissa};
	\item per ciascun diagramma \emph{\textbf{U}nified \textbf{M}odeling \textbf{L}anguage} (\acrshort{UML}) è utilizzato lo standard 2.0.
\end{itemize}
