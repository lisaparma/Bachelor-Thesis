%% INIZIO CONTENUTO TESI
\mainmatter

\chapter{Introduzione}\label{chap:intro}

\section{Organizzazione dell'elaborato}
Questo elaborato è organizzato come segue: nel Capitolo~\ref{chap:intro} viene
descritta la struttura del documento e le convenzioni adottate nella stesura
dello stesso, mentre nel Capitolo~\ref{chap:company} viene introdotta l'azienda
ospitante e il metodo di lavoro seguito al suo interno. Nel
Capito~\ref{chap:project} viene descritto in progetto e il mio contributo ad
esso e nel capitolo successivo, il~\ref{chap:tec}, le tecnologie che sono state
studiate e adottate durante il mio stage in quanto erano da me poco conosciute 
e/o mai adoperate. Successivamente, nel
Capitolo~\ref{chap:requirements} sono elencati i requisiti che l'azienda ha
definito per il prodotto finale, mentre nel Capitolo~\ref{chap:design} sono
spiegate le scelte implementative e di design dell'applicazione. Infine, nel Capitolo~\ref{chap:tests}
vengono descritte le fasi di verifica e validazione del prodotto, mentre nel
Capitolo~\ref{chap:conclusions} è possibile trovare delle mie considerazioni
sia sull'applicazione prodotta, che sul lavoro svolto e il periodo di stage in
generale.

\section{Convenzioni adottate}
Nella stesura del presente documento sono state adottate le seguenti convenzioni
tipografiche:
\begin{itemize}
	\item la prima occorrenza di termini tecnici, ambigui o di acronimi verrà marcata in corsivo	e con una G a pedice, in modo da indicare la sua presenza nel glossario riportato in fondo a questo documento;
	\item i termini tecnici in lingua inglese non traducibili e non presenti nel glossario, verranno evidenziati in corsivo;
	\item la prima occorrenza di un acronimo viene riportata con la dicitura estesa, in
	corsivo, e le lettere che compongono l’acronimo vengono riportate in grassetto. Ogni successiva occorrenza presenterà solo la forma ridotta;
	\item le parole chiave presenti in ciascun paragrafo saranno marcate in grassetto, in modo da consentirne una facile individuazione;
	\item ogni termine corrispondente a nomi di file, componenti dell’architettura o codice verrà marcato utilizzando un font a larghezza fissa;
	\item per ciascun diagramma Unified Modeling Language (UML) è utilizzato lo
	standard 2.0.
\end{itemize}
