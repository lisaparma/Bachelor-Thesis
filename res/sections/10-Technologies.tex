% Tutto quello riguardante il progetto andrà in questo capitolo

\chapter{Tecnologie utilizzate}\label{chap:tec}
Per lo sviluppo dell'applicativo Teamwork ho utilizzato diverse tecnologie da me prima mai adoperate. Prima della progettazione e dello sviluppo di essa ho svolto un lavoro di studio per prepararmi al meglio al loro utilizzo. Esse sono:

\section{Ambiente di sviluppo}
\subsection{Docker}
\begin{figure}[H] 
	\centering
	\includegraphics[scale=0.4]{docker}
	\caption{Logo Docker}
\end{figure}
Docker permette di automatizzare lo sviluppo di applicazione dentro a determinati container software, in modo da fornire una virtualizzazione a livello del sistema operativo Linux senza che sia necessario istanziare delle macchine virtuali pienamente operative.
I container sono assimilabili a delle macchine virtuali modulari che hanno il pregio di essere molto leggere: in questo modo risulta più facile  creare, distribuire, copiare e spostare i container da un ambiente all'altro.
Ogni container è un processo isolato dagli altri container e dal sistema ospite stesso:  ogni volta che viene mandato in esecuzione si avrà un ambiente pulito, con le caratteristiche desiderate. \\
Per sviluppare l'applicazione Teamwork è stato utilizato Docker 18.03 Community Edition per Ubuntu Bionic. In questo modo si è potuto caricare nelle macchine locali un container di Zimbra e del core Zextras con la zimlet OpenChat caricata così da sviluppare l'applicativo in un ambiente sicuro e pulito.

\subsection{IntelliJ IDEA}
\begin{figure}[H] 
	\centering
	\includegraphics[scale=0.3]{intellij-idea}
	\caption{Logo IntelliJ IDEA}
\end{figure}
IntelliJ IDEA è un Integrated Development Environment (IDE) multi-linguaggio
e multi-piattaforma. ...

\subsection{Expo}
\begin{figure}[H] 
	\centering
	\includegraphics[scale=0.08]{expo}
	\caption{Logo Expo}
\end{figure}

\section{Linguaggi, librerie e strumenti di programmazione}
\subsection{React Native}
\begin{figure}[H] 
	\centering
	\includegraphics[scale=0.17]{React}
	\caption{Logo React Native}
\end{figure}
\subsection{Typescript}
\begin{figure}[H] 
	\centering
	\includegraphics[scale=0.1]{ts}
	\caption{Logo TypeScript}
\end{figure}
\subsection{TSLint}
\begin{figure}[H] 
	\centering
	\includegraphics[scale=0.5]{TSlint}
	\caption{Logo TS Lint}
\end{figure}
\subsection{Redux}
\begin{figure}[H] 
	\centering
	\includegraphics[scale=0.08]{redux}
	\caption{Logo Redux}
\end{figure}
 libreria Javascript per la gestione semplificata dello stato delle applicazioni web.
 
\section{Strumenti di verifica e valutazione}
\subsection{Genymotion}
\begin{figure}[H] 
	\centering
	\includegraphics[scale=0.25]{genymotion2}
	\caption{Logo Genymotion}
	
\end{figure}
\subsection{React Native Debugger}
\begin{figure}[H] 
	\centering
	\includegraphics[scale=0.3]{react-native-debugger}
	\caption{Logo React Native Debugger}
\end{figure}
\subsection{Jest}
\begin{figure}[H] 
	\centering
	\includegraphics[scale=0.25]{jest}
	\caption{Logo Jest}
\end{figure}
