% VERIFICA E VALIDAZIONE

\chapter{Verifica e Validazione}\label{chap:tests}
\section{Linter}
Come strumento di verifica statica del codice è stato utilizzato TSLint, descritto nella sezione \ref{subsec:tslint} di questo elaborato. Esso si presenta come uno strumento che segnala errori di programmazione, errori stilistici e segnala costrutti sospetti nel codice in base a determinati criteri personalizzabili durante la codifica in programmi scritti in TypeScript.\\
In particolare è stato utilizzato per rendere conforme il codice scritto alle convenzioni utilizzate in azienda. Degli esempi sono:
\begin{itemize}
	\item non lo so
\end{itemize}
\section{Jest}

\section{Test di sistema e collaudo}
Per accertare la copertura dei requisiti si è ricorso a delle prove pratiche del corretto funzionamento dell’applicazione. Questa operazione è stata svolta sia come strumento per la verifica delle operazioni durante lo sviluppo del prodotto software che come operazione di collaudo per assicurare il corretto funzionamento del prodotto una volta implementate tutte le funzioni richieste. \\
Grazie al framework per lo sviluppo utilizzato, Expo, è stato facile testare fin dalle prime implementazioni di funzionalità il prodotto sia attraverso emulatori che su device fisici.
\subsection{Emulatori}
%Durante lo sviluppo del codice si è ritenuto opportuno avere sempre a disposizione un emulatore per assicurarsi che ogni passo sia ...Utilizzo di emulatore Genymotion per testare il funzionamento su schermi e risoluzioni differenti.
\subsection{Device fisici}
%tilizzo di dispositivo android fornito dall’azienda. 

\subsection{Beta testing}