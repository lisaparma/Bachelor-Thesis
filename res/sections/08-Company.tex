%% AZIENDA OSPITANTE
\chapter{Azienda ospitante}\label{chap:company}

\section{Profilo dell'azienda}
\begin{figure}[H] 
	\centering
	\includegraphics[scale=0.3]{zextras}
	\caption{Logo Zextras s.r.l.}
	\label{fig:logoZextras}
\end{figure}
L'attività di stage descritta in questo elaborato è stata svolta presso l'azienda Zextras s.r.l. nella sede principale di Torri Di Quartesolo (VI). \\
La società è nata nel 2011 con l'obbiettivo di espandere le possibilità della \mmh{collaboration suite Zimbra}{cos'è? dico qui due paroline anche?}.
In pochi anni è riuscita a rendere la propria suite di prodotti l'estensione professionale per Zimbra Open Source più avanzata tra le soluzioni di collaborazione sul mercato. I loro prodotti sono riconosciuti come ottimali anche dalla \mmh{community Zimbra}{Avrà u nome no?} tanto che è in corso una collaborazione per lo sviluppo di alcuni prodotti open source e per Zimbra Suite Plus.
Ad oggi presenta uffici secondari in Francia, Brasile, Russia e negli Stati Uniti e possiede partner in tutto il mondo.\\


\subsection{Zimbra}
\begin{figure}[H] 
	\centering
	\includegraphics[scale=0.2]{zimbra}
	\caption{Logo Zimbra}
	\label{fig:logoZimbra}
\end{figure}
Zimbra è un server di workgroup open source che mette a disposizione una suite di software collaborativi che consentono di condividere documenti e attività. 
Di base essa offre i seguenti servizi:
\begin{itemize}
	\item[•] configurazione personalizzata,
	\item[•] gestione della posta elettronica,
	\item[•] rubriche, calendari e condivisione di file,
	\item[•] chat e chiamate vocali,
	\item[•] integrazione con i canali web
	\item[•] privacy e alti livelli di sicurezza
	\item[•] disponibilità per i dispositivi mobili: oltre che mediante client Zimbra Web e attraverso i client di posta elettronica tradizionale, è possibile accedere alle e-mail, ai calendari e alle altre offerte da dispositivi mobile. 
\end{itemize}
Le funzionalità di Zimbra possono essere facilmente estese grazie alla possibilità di creare ed aggiungere degli add-on chiamati Zimlet. Esse sono delle integrazioni che permettono di personalizzare i servizi ed integrarli con servizi web e applicazioni di terzi.
Zimbra è disponibile in due versioni: 
\begin{itemize}
	\item Zimbra Open Source Edition: soluzione completamente open source ma con disponibili solo le funzionalità standard;
	\item Zimbra Network Edition: soluzione completa di tutte le funzionalità sia per l'amministratore di sistema sia per l'utente finale.
\end{itemize}


\subsection{Zextras Suite}
Zextras Suite è un Add-On per Zimbra Collaboration: i suoi prodotti sono progettati per espandere le funzioni di Zimbra Open Source Edition in maniera a se stante rispetto i moduli Zimbra Network Edition. Infatti Zextras Suite non è distribuito assieme ad alcun binario o sorgente sotto il copyright di Zimbra. \\
Le principali innovazioni che sono state portate nel mondo Zimbra riguardano la sicurezza dei dati, la mobilità e la gestione dello storage.\\
La suite comprende i seguenti prodotti:
	\begin{itemize}
		\item Zextras Backup: un software che permette il backup in realtime e il ripristino per tutti i dati Zimbra;
		\item Zextras Mobile: per sincronizzare le email, i contatti, gli eventi e ogni task con qualsiasi device mobile tramite Exchange ActiveSync;
		\item Zextras Powerstore: per ottimizzare i volumi dei dati Zimbra e risparmiare spazio attraverso la compressione e la deduplicazione;
		\item Zextras Admin: per monitorare gli utenti e le funzionalità di Zimbra, Zextras Suite e ogni altro Zimlet;
		\item Zextras Chat: piattaforma client/server integrata in Zimbra per la messaggistica istantanea e le videochat.
	\end{itemize}

\begin{figure}[H] 
	\centering
	\includegraphics[scale=0.3]{zextras_2018}
	\caption{Rappresentazione Zextras suite}
	\label{fig:modulizextras}
\end{figure}
Come rappresentato in figura~\ref{fig:modulizextras}, Zextras suite è un'estensione modulare per Zimbra Open Source che può essere applicata secondo le proprie esigenze. Proprio per questo risulta la migliore scelta per un uso professionale di Zimbra.

\section{Metodo di lavoro}
La necessità di sviluppare progetti particolarmente complessi ma di mantenere flessibilità e trasparenza ha portato l'azienda a preferire l'utilizzo di un ciclo di sviluppo software di tipo agile rispetto a metodi di management più tradizionali che spesso rischiano di rallentare e disperdere le risorse.
La filosofia dello sviluppo agile si basa su quattro pilastri fondamentali che caratterizzano. \\ Esse sono:
\begin{itemize}
	\item gli individui e le interazioni più che i processi e gli strumenti;
	\item  il software funzionante più che la documentazione esaustiva;
	\item la collaborazione col cliente più che la negoziazione dei contratti;
	\item  rispondere al cambiamento più che seguire un piano.
\end{itemize}
Questi pilastri sono ampiamente condivisi dalla filosofia aziendale che basa lo sviluppo dei propri software sulla continua interazione con gli stakeholder e con i clienti.

\section{Scrum}
In particolare Zextras utilizza il metodo Scrum che risulta uno dei più diffusi ed è particolarmente adatto a progetti complessi ed innovativi.\\
Scrum è un ciclo di sviluppo iterativo nel quale ogni iterazione viene definita Sprint: esso è un periodo di tempo che solitamente dura da 1 a 4 settimane nel quale viene prefissata una  lista di requisiti e funzionalità che devono essere implementate nel periodo deciso.
Questa lista di funzionalità e requisiti viene definita prima in una Product Backlog, documento che contiene tutti i requisiti necessari per la realizzazione del progetto, poi, periodicamente, in una Sprint Backlog, documento che definisce tutti i task da completare nei singoli sprint. 
In azienda si lavora su sprint di due settimane in cui la fine di ogni sprint coincide con una release del software.
Per gestire al meglio gli sprint e le backlog di ogni progetto vengono eseguite delle riunioni periodiche:
\begin{itemize}
	\item Sprint Planning: riunione precedente l'inizio del progetto in cui si stila la Product Backlog e si determina il numero e la durata degli sprint in base al tempo a disposizione e agli obiettivi. Inoltre viene definito la Sprint Backlog per il primo sprint;
	\item Daily Scrum: breve confronto giornaliero che permette di sincronizzare i lavoro di tutto il team e di affrontare tempestivamente i problemi riscontrati;
	\item Sprint Review: revisione al termine di ogni sprint che serve a valutare se gli obiettivi prefissati sono stati esaustivamente completati o se bisogna ricalcolare il lavoro per il prossimo sprint;
\end{itemize}



\subsection{Strumenti a supporto di processi e servizi}
\subsubsection{BitBucket}
Per il versionamento del software Uqido S.r.l. utilizza Git, affidandosi a GitHub
e a BitBucket come host (entrambi gratuiti) per i progetti in corso
di sviluppo.
\subsubsection{Jira}
..Per favorire il corretto funzionamento dei processi adottati dall’azienda, essa
utilizza BaseCamp per il project management. BaseCamp è un task manager
\subsubsection{Bamboo}
\subsubsection{Confluence}
..Per la realizzazione aziendale di documenti inerenti ai progetti viene utilizza

